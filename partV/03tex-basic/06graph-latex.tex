
% latex nombre_archivo.tex
% dvipdf nombre_archivo.dvi
\documentclass{article}
\usepackage{pstricks}
\usepackage{pst-plot} 
\usepackage{xcolor}

\title{My document about graphics}
\author{J. A. Orduz-Ducuara}
\begin{document}
\maketitle
\section{My first document on graphics with \LaTeX}
In this document we will explore 
the \verb|pstricks| package.
We study some options, the reader has to check the information 
and more
(for instance in my mac: 
/usr/local/texlive/2017/texmf-dist/doc/generic/pstricks
)
The general form to implement the 
tool is:\\

\begin{verbatim}
\begin{pspicture}([Xmin],[Ymin])([Xmax],[Ymax])
...[COMANDOS]
\end{pspicture}
\end{verbatim}

\subsection{How can I create a vertical line?}
\begin{pspicture}(6,3)
  \psline[linestyle=dashed](6, 0)(6, 3)
 \end{pspicture}
 
\noindent where we use
with \verb|\begin{pspicture}(x0, y0)(x1, y1)|
is the size of the picture. 
And to create a line
\begin{verbatim}
\begin{pspicture}(3,3)
  \psline[linestyle=dashed](0, 0)(0, 3)
 \end{pspicture}
\end{verbatim}

We can change the options and introduce other:




or we can explore other line type:
\begin{pspicture}(4,3) \psset{arrowscale=2,linewidth=1pt}
\psline{]-[}(4,0)
\psline{)-(}(0,1)(4,1)
\psline{)->}(0,2)(4,2)
\psline{]->>}(0,3)(4,3)
\end{pspicture}





We should plot a complex graphics.\\
\begin{pspicture}(3,3)
  \psline{<->}(0, 0)(3, 0)
 \end{pspicture}

\begin{verbatim}
\begin{pspicture}(3,3)
  \psline{<->}(0, 0)(3, 0)
 \end{pspicture}
\end{verbatim}


or \\
\begin{pspicture}(3,3)
  \psline{<->}(0, 0)(3, 0)
\end{pspicture}

\begin{verbatim}
\begin{pspicture}(3,3)
  \psline{<->}(0, 0)(3, 0)
\end{pspicture}
\end{verbatim}

and\\

\begin{pspicture}(3,3)
  \psline{<->}(0,3)(0, 0)(3, 0)
 \end{pspicture}


\begin{verbatim}
\begin{pspicture}(3,3)
  \psline{<->}(0,3)(0, 0)(3, 0)
 \end{pspicture}
\end{verbatim}

\subsection{To create circles}
  

\begin{pspicture}(0.5\linewidth,4)
  \pscircle[fillstyle=vlines,hatchangle=0,hatchsep=0.6pt,%
    hatchwidth=1pt,hatchwidthinc=0.3pt,hatchangle=90,
    hatchcolor=red](2,2){2}
  \pscircle[fillstyle=vlines,hatchangle=0,hatchsep=0.6pt,%
    hatchwidth=1pt,hatchwidthinc=0.3pt,hatchangle=-45,
    hatchcolor=green](7,2){2}
  \pscircle[fillstyle=hlines,hatchangle=0,hatchsep=0.6pt,%
    hatchwidth=1pt,hatchwidthinc=0.3pt,hatchangle=45,
                                                 hatchcolor=blue](12,2){2}
\end{pspicture}


\begin{pspicture}(-5, -5)(5, 5)
\pscircle[linecolor=blue,linestyle=dashed](1,1.5){1}
\pscircle[linecolor=blue,linestyle=dashed](2,1.5){1}
\pscircle[linecolor=blue,linestyle=dashed](3,1.5){1}
\pscircle[linecolor=blue,linestyle=dashed](4,1.5){1}
\pscircle[linecolor=blue,linestyle=dashed](5,1.5){1}
\end{pspicture}

where we use
\begin{verbatim}
\begin{pspicture}(0, 0)(0, 5)
\pscircle[linecolor=blue,linestyle=dashed](1,1.5){1}
\pscircle[linecolor=blue,linestyle=dashed](2,1.5){1}
\pscircle[linecolor=blue,linestyle=dashed](3,1.5){1}
\pscircle[linecolor=blue,linestyle=dashed](4,1.5){1}
\pscircle[linecolor=blue,linestyle=dashed](5,1.5){1}
\end{pspicture}
\end{verbatim}

\subsection{Complex figures}

\begin{pspicture}[showgrid](-4,-5)(4,7)
    \psframe(-3,4)(3,5)
    \psccurve(-3,2)(-1,2)(-1,3)(-1.5,2.5)
    \pspolygon(1,2)(3,2)(2,3)
    \pswedge(-2,0){1}{0}{270}
    \psellipticwedge(2,0)(1,0.5){0}{270}
    \pscircle(-2,-3){1}
    \psellipse(2,-3)(1,0.5)
\rput(0,6.5){This is a message}
\end{pspicture}



\begin{verbatim}
\begin{pspicture}[showgrid](-4,-5)(4,7)
    \psframe(-3,4)(3,5)
    \psccurve(-3,2)(-1,2)(-1,3)(-1.5,2.5)
    \pspolygon(1,2)(3,2)(2,3)
    \pswedge(-2,0){1}{0}{270}
    \psellipticwedge(2,0)(1,0.5){0}{270}
    \pscircle(-2,-3){1}
    \psellipse(2,-3)(1,0.5)
\rput(0,6.5){This is a message}
\end{pspicture}
\end{verbatim}

\begin{pspicture}(\linewidth,3)
\psframe[fillstyle=hlines,hatchangle=0,hatchangle=-60,%
    hatchwidth=1pt,hatchsep=0.5pt,hatchsepinc=0.1pt,
    hatchcolor=blue](\linewidth,3)
\end{pspicture}

\noindent
\begin{verbatim}
\begin{pspicture}[showgrid](-4,-5)(4,7)
    \psframe(-3,4)(3,5)
    \psccurve(-3,2)(-1,2)(-1,3)(-1.5,2.5)
    \pspolygon(1,2)(3,2)(2,3)
    \pswedge(-2,0){1}{0}{270}
    \psellipticwedge(2,0)(1,0.5){0}{270}
    \pscircle(-2,-3){1}
    \psellipse(2,-3)(1,0.5)
\rput(0,6.5){This is a message}
\end{pspicture}
\end{verbatim}

\begin{pspicture}[showgrid](-4,2)(4,7)
\pscustom[fillstyle=eofill, fillcolor=red, opacity = 0.4, linecolor = white]
{ 
  \psframe(-3, 6)(3,7)
}
\pscustom[fillstyle=eofill, fillcolor=blue, opacity = 0.9]
{ 
  \psframe(-3, 4)(3, 5)
  }
\end{pspicture}



\begin{verbatim}
\begin{pspicture}[showgrid](-4,2)(4,7)
\pscustom[fillstyle=eofill, fillcolor=red, opacity = 0.4, linecolor = white]
{ 
  \psframe(-3, 6)(3,7)
}
\pscustom[fillstyle=eofill, fillcolor=blue, opacity = 0.9]
{ 
  \psframe(-3, 4)(3, 5)
  }
\end{pspicture}
\end{verbatim}

\begin{pspicture}[showgrid](-5, -5)(5, 5)
\pscurve[showpoints=true, dotsize=10pt, dotstyle=o, fillcolor=red]
(0, 0)(1, 2)(2, 3)(2, -5)(0 ,0)
\end{pspicture}

\begin{verbatim}
\begin{pspicture}(3, 5)
\pscurve[showpoints=true, dotsize=20pt, dotstyle=o, fillcolor=red]
(0, 0)(1, 2)(2, 3)(-2, -5)
\end{pspicture}
\end{verbatim}


\begin{pspicture}[showgrid](-3,-3)(3, 3)
\psbezier[showpoints=true, dotsize=20pt, dotstyle=o, fillcolor=red]
(2, -2)(-2,-2)(0,2)(3, 3)
\end{pspicture}


\begin{verbatim}
\begin{pspicture}(3,3)
\psccurve[showpoints=true,
dotsize=20pt,
dotstyle=o,
fillcolor=red]
(0,0)(1,1)(1,2)(2,1)(1,0)
\end{pspicture}
\end{verbatim}


\section{Functions on {\LaTeX}}

\begin{pspicture}(-7.5,-3)(7.5,3)
  \psaxes[labels=none](0,0)(-7,-2)(7,2)    % establece los ejes
   \psplot[linecolor=blue, linewidth=1.5pt]
     {-7}{7}{x 0.01745329252 div sin}      
   
 \psplot[linecolor=green, linewidth=1.5pt]
     {-7}{7}{x 0.01745329252 div cos}          
   \uput[45](3.1415926,0){$\pi$}           % estas son las etiquetas
   \uput[90](-1.570796,0){$-\pi/2$}        % \uput es una caja posicionada en [angulo]
   \uput[-90](1.570796,0){$\pi/2$}         % relativo a la coordenada (x,y)
   \uput[-135](-3.1415926,0){$-\pi$}       % y poniendo el { contenido } dentro de la caja
   \psline[linewidth=1pt,linecolor=red,linestyle=dotted]%   % línea punteada roja
     (1.57079632,1)(1.57079632,0) 
   \psline[linewidth=1pt,linecolor=red,linestyle=dotted]%
     (-1.57079632,-1)(-1.57079632,0) 
 \end{pspicture}


\begin{verbatim}
\begin{pspicture}(-7.5,-3)(7.5,3)
  \psaxes[labels=none, labelwidth=13.5pt](0,0)(-7,-2)(7,2)    % establece los ejes
   \psplot[linecolor=blue, linewidth=1.5pt]% grafica una onda sinusoidal
     {-7}{7}{x 0.01745329252 div sin}      % observar la expresión en 
                                           % Notación Polaca Inversa
   \uput[45](3.1415926,0){$\pi$}           % estas son las etiquetas
   \uput[90](-1.570796,0){$-\pi/2$}        % \uput es una caja posicionada en [angulo]
   \uput[-90](1.570796,0){$\pi/2$}         % relativo a la coordenada (x,y)
   \uput[-135](-3.1415926,0){$-\pi$}       % y poniendo el { contenido } dentro 
                                           % de la caja
   \psline[linewidth=1pt,linecolor=red,linestyle=dotted]%   % línea punteada roja
     (1.57079632,1)(1.57079632,0) 
   \psline[linewidth=1pt,linecolor=red,linestyle=dotted]%
     (-1.57079632,-1)(-1.57079632,0) 
 \end{pspicture}
\end{verbatim}


\end{document}
